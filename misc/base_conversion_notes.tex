\documentclass{article}
\usepackage{amsmath}
\usepackage{amsthm}
\usepackage{hyperref}

\theoremstyle{plain}
\newtheorem{lemma}{Lemma}
\newtheorem{proposition}[lemma]{Proposition}
\newtheorem{fact}[lemma]{Fact}

\theoremstyle{definition}
\newtheorem{definition}[lemma]{Definition}
\newtheorem{example}[lemma]{Example}

\title{Roundtrip conversions between floating-point formats}
\author{Mark Dickinson}

\begin{document}

\maketitle

This note is a followup from a discussion with Rick Regan in the comments
section of a \href{http://stackoverflow.com/a/35708911/270986}{Stack Overflow
  question}[\ref{so1}]. The aim is to give a short proof of the condition for
roundtripping of conversions between two floating-point formats.

Note: all results presented here are well-known. The proofs of the main
propositions are, as far as I know, original.

\begin{section}{Preliminaries}

We work with an idealised form of floating-point that ignores exponent bounds,
NaNs, infinities and zeros. Roundtripping for negative floats behaves entirely
analogously to that for positive floats, so for simplicity we only consider
positive floats in what follows.  We consider only round-to-nearest rounding
modes between floating-point formats, and we make no assumptions about which
direction ties round in, so our results are equally applicable to the
round-ties-to-even and the round-ties-away-from-zero rounding modes, for
example.

To be precise, here's what we'll mean by a floating-point number in what
follows.

\begin{definition}
  Given integers $p \ge 1$ and $B \ge 2$, a \textbf{precision-$p$ base-$B$
    floating-point number} is a positive rational number $x$ that can be
  expressed in the form $x = m B^e$ for some integers $m$ and $e$, with $0 \le
  m < B^p$.
\end{definition}

We're interested in what happens when we convert a number in one floating-point
format to another format and back again, with each of the two conversions
rounding to nearest. Specifically, we want to give necessary and sufficient
conditions on the two formats for this double conversion to recover the number
we started with.

For the remainder of this note, we fix two floating-point formats. Choose
integers $p \ge 1$, $q \ge 1$, $B \ge 2$ and $D\ge 2$. Write $\mathbf B$ for
the set of all precision-$p$ base-$B$ floats, and $\mathbf D$ for the set of
all precision-$q$ base-$D$ floats.

\begin{definition}
  We say that a precision-$p$ base-$B$ float $x$ in $\mathbf B$
  \textbf{roundtrips through $\mathbf D$} if the nearest precision-$q$ base-$D$
  floating-point value to $x$ rounds back to $x$. Or in other words, let $y$ be
  the closest element of $\mathbf D$ to $x$. Then we require that $x$ is the
  closest element of $\mathbf B$ to $y$. In the event that $x$ is exactly
  halfway between two elements of $\mathbf D$, we require that \emph{both}
  those elements round back to $x$.

  We'll also say that the \emph{format} $\mathbf B$ roundtrips through $\mathbf
  D$ if every element of $\mathbf B$ roundtrips through $\mathbf D$.
\end{definition}

\end{section}

\begin{section}{Necessary and sufficient conditions}

Now we can state the main propositions. There's a simple sufficient condition
for roundtripping.

\begin{proposition}
  \label{roundtrip_sufficient_condition}
  If $B^p \le D^{q-1}$ then the format $\mathbf B$ roundtrips through $\mathbf
  D$.
\end{proposition}

Under one additional hypothesis, this condition is also necessary.

\begin{proposition}
  \label{roundtrip_necessary_condition}
  If the format $\mathbf B$ roundtrips through $\mathbf D$, \emph{and} $B$ and
  $D$ are not powers of a common base, then $B^p \le D^{q-1}$.
\end{proposition}

The \emph{powers of a common base} condition excludes cases like $B = 4$ and $D
= 8$, or $B = 25$ and $D = 5$.

To prove the first proposition, we need a pair of lemmas relating to the
spacing between adjacent floating-point numbers.

\begin{lemma}
  \label{close_implies_round}
  Suppose that $x$ is an element of $\mathbf B$, and that $B^{e-1} < x \le B^e$
  for some integer $e$. Then any positive rational number $y$ satisfying $|x -
  y| < \frac12 B^{e-p}$ rounds back to $x$.
\end{lemma}

\begin{proof}
  This follows from the observation that within the closed interval $[B^{e-1},
    B^e]$, adjacent floats in $\mathbf B$ are spaced exactly $B^{e-p}$ apart
  from one another, so if $y$ is within the interval $[B^{e-1}, B^e]$ then $x$
  must be the closest element of $\mathbf B$ to $y$, and there's nothing more
  to show. The only way for $y$ to \emph{not} be in the interval is if $x=B^e$
  and $B^e < y < B^e + \frac12 B^{e-p}$. But then the next element of $\mathbf
  B$ up from $x$ is $B^e + B^{e-p+1}$, and again $x$ is the closest element
  of $\mathbf B$ to $y$.
\end{proof}

Note that in the above, the strictness of the inequalities in our condition on
$e$ is important: if we replace the condition $B^{e-1} < x \le B^e$ with
$B^{e-1}\le x < B^e$, the conclusion is no longer valid, since there will be
values $y$ just below $B^{e-1}$ which round to the next float down from
$B^{e-1}$.

\begin{lemma}
  \label{round_implies_close}
  Suppose that $x$ is any positive rational number, that $f$ is the unique
  integer such that $D^{f-1} < x \le D^f$, and that $y$ is a closest element of
  $\mathbf D$ to $x$, choosing either possibility in the case of a tie. Then
  $|x - y| \le \frac12 D^{f-q}$.
\end{lemma}

\begin{proof}
  This follows from the fact that $D^{f-1} \le y \le D^f$, and that successive
  floats in $\mathbf D$ are spaced exactly $D^{f-q}$ apart in this interval.
\end{proof}

The proof of the sufficient condition is now straightforward.

\begin{proof}[Proof of Proposition \ref{roundtrip_sufficient_condition}]
  Let $x$ be any precision-$p$ base-$B$ floating-point number, and let $y$ be a
  closest precision-$q$ base-$D$ floating-point number to $x$ (picking either
  one in the case of a tie).  Choose integers $e$ and $f$ such that $B^{e-1} <
  x \le B^e$ and $D^{f-1} < x \le D^f$. Then we have:
  \begin{align*}
    |x - y|
    &\le \frac12 D^{f-q} && \text{by Lemma \ref{round_implies_close}}\\
    &= \frac12 D^{f-1} D^{1-q} \\
    &< \frac12 x D^{1-q} && \text{by choice of $f$}\\
    &\le \frac12 x B^{-p} &&
                   \text{from the assumption that $B^p \le D^{q-1}$}\\
    &\le \frac12 B^e B^{-p} && \text{by choice of $e$}\\
    &= \frac12 B^{e-p} \\
  \end{align*}
  So $|x - y| < \frac12 B^{e-p}$, and it follows from Lemma
  \ref{close_implies_round} that $y$ rounds back to $x$.
\end{proof}

For the necessary condition, we'll need the following result from elementary
number theory.

\begin{fact}\label{dense_power_ratios}
  Suppose that $B$ and $D$ are integers larger than $1$, and that $B$ and $D$
  are not powers of a common base (or equivalently, $log(B) / log(D)$ is not
  rational). Then numbers of the form $B^e / D^f$ are \emph{dense} in the
  positive reals: any open subinterval $(a, b)$ of the positive reals contains
  at least one such number.
\end{fact}

\begin{proof}
  Given our target interval $(a, b)$, it's enough to show that the interval
  $(1, b/a)$ contains at least one such number, $\gamma$ say.  Given such a
  $\gamma$, there must be a power of $\gamma$ within $(a, b)$. (Take $k =
  \lfloor \log_{\gamma}a\rfloor$, then $\gamma^k \le a < \gamma^{k+1}$, and
  $\gamma^{k+1} < (b/a)\gamma^k \le (b/a)a = b$. So $\gamma^{k+1}$ lies in the
  interval $(a, b)$.)

  So it's enough to prove the fact in the case $a=1$. Suppose, for a
  contradiction, that the interval $(1, b)$ contains no elements of the form
  $B^e / D^f$. Then the set of all elements of the form $B^e / D^f$ is discrete
  (if $\gamma$ is any such element, then there can be no other elements within
  the interval $(\gamma/b, \gamma b)$), and so amongst all elements of the form
  $B^e / D^f$ larger than $1$ there must be a smallest such element; call it
  $C$. Now \emph{all} elements of the form $B^e/D^f$ must be integral powers of
  $C$ (let $\gamma$ be any such element, then $1 \le \gamma /
  C^{\lfloor\log_C(\gamma)\rfloor} < C$; this contradicts the choice of $C$ as
  the smallest element \emph{unless} we have $1 = \gamma /
  C^{\lfloor\log_C(\gamma)\rfloor}$), so we have $B=C^m$ and $D=C^n$ for some
  positive integers $m$ and $n$. Remembering that $C$ is rational, it follows
  that $C$ must be an integer, hence that $B$ and $D$ are powers of a common
  base, contradicting our assumptions.
\end{proof}

Now we can prove the necessary condition. We'll show the contrapositive, namely
that if $B^p > D^{q-1}$ then there's some element $x$ of $\mathbf B$ that fails
to roundtrip through $\mathbf D$. Before we embark on the proof proper, let's
sketch the main idea. To find our roundtrip counterexample $x$, we look for a
region of the positive reals where the gap between successive elements of
$\mathbf B$ is \emph{smaller} than the gap between successive elements of
$\mathbf D$. In relative terms, the gap between successive elements of $\mathbf
B$ is smallest just \emph{before} a power of $B$, while the gap between
successive elements of $\mathbf D$ is largest just \emph{after} a power of
$D$. So if there's an $x$ that fails to roundtrip, a good place to look for it
would be in an interval $[D^f, B^e]$ where $D^f$ and $B^e$ are very close to
one another. In such an interval, the gap between successive elements of
$\mathbf D$ is $D^{f-q+1}$, while the gap between successive elements of
$\mathbf B$ is $B^{e-p}$. Our hypothesis that $B^p > D^{q-1}$ implies that $1 <
B^p / D^{q-1}$, so by Fact \ref{dense_power_ratios} we should be able to find
$e$ and $f$ so that $1 < B^e / D^f < B^p / D^{q-1}$. Then our gaps satisfy
$B^{e-p} < D^{f-q+1}$, as required.

Making this idea rigorous turns out to be a bit fiddly: just because the gaps
work out nicely in a particular interval, that doesn't guarantee that we can
find a suitable $x$ in that interval: the interval might be too small to
contain \emph{any} suitable values $x$.

It turns out that we can always find a roundtrip failure $x$ of the form
$x=B^e$: by careful choice of $e$ and $f$, we'll be able to find an interval
$[D^f, B^e]$ so that
\begin{enumerate}
  \item $B^e$ is sufficiently close to $D^f$ that $B^e$ rounds down to $D^f$
    when converting from $\mathbf B$ to $\mathbf D$, but
  \item $D^f$ is sufficiently far away from $B^e$ that $D^f$ rounds down to
    some value \emph{smaller} than $B^e$ when converting back from $\mathbf D$
    to $\mathbf B$.
\end{enumerate}
Then $x=B^e$ gives us our roundtrip failure. Here's the formal proof.

\begin{proof}[Proof of Proposition \ref{roundtrip_necessary_condition}]
  We prove the contrapositive: that if $B^p > D^{q-1}$ then there's at least
  one element $x$ of $\mathbf B$ that fails to roundtrip through $\mathbf D$.
  We'll find such an $x$ of the form $x=B^e$, and show that there exist
  integers $e$ and $f$ such that $B^e$ rounds down to $D^f$, but $D^f$ rounds
  down again to some value smaller than $B^e$.

  For $B^e$ to round down to $D^f$, we need:
  $$D^f < B^e < D^f + \frac12 D^{f-q+1},$$ while for $D^f$ to round down to
  some value strictly smaller than $B^e$, we need
  $$D^f < B^e - \frac12 B^{e-p}.$$
  Rearranging the inequalities above gives the condition
  $$\frac{1}{1 - \frac12 B^{-p}} < B^e / D^f < 1 + \frac12 D^{1-q}.$$ So
  \emph{if} we can find integer exponents $e$ and $f$ such that the above
  holds, then $x=B^e$ gives us a counterexample to roundtripping.

  But from Fact \ref{dense_power_ratios}, we can find a number of the form $B^e
  / D^f$ in \emph{any} open subinterval of the positive reals. The only thing
  we need to check is that the two inequalities above are compatible; that is,
  that
  $$\frac{1}{1 - \frac12 B^{-p}} < 1 + \frac12 D^{1-q},$$
  so that the lower and upper bounds really do form a subinterval of the reals.
  But the above inequality can be rearranged to the equivalent inequality
  $$\frac12 < B^{p} - D^{q-1},$$
  which is true from our assumption that $B^p > D^{q-1}$.

  So this completes the proof: find $e$ and $f$ satisfying the above
  inequality, and then $x = B^e$ will fail to roundtrip.
\end{proof}

\begin{example}
  Consider precision-$53$ binary (as used in double-precision IEEE 754
  floating-point, for example). We have
  $$2^{53} \le 10^{17 - 1}$$ and it follows that conversion from a finite IEEE
  754 binary64 float to a decimal string with $17$ significant digits produces
  a value that rounds back to the original float. $16$ significant digits are
  insufficient.
\end{example}

\begin{example}
  Consider conversions from $10$-bit binary to $4$-digit decimal and back
  again.  The necessary and sufficient condition for roundtripping is that
  $$2^{10} \le 10^{4-1},$$ which is false (but only just). So roundtripping
  does not hold, and we should be able to find an explicit $10$-bit binary
  float which fails to round-trip through $4$-digit decimal.

  Following the above proof, we look for exponents $e$ and $f$ such that
  $$\frac{1}{1 - \frac12 2^{-10}} < \frac{2^e}{10^f} < 1 + \frac12 10^{-3}$$
  which simplies to the condition
  $$\frac{2048}{2047} < \frac{2^e}{10^f} < \frac{2001}{2000}.$$ The smallest
  positive exponents satisfying this pair of inequalities are $f=1929$ and
  $e=6408$ (in fact, this $f$ is the \emph{only} positive value smaller than
  $10000$ for which there's a solution). So $x = 2^{6408}$ fails to roundtrip:
  the nearest $4$-digit decimal to $x$ is $10^{1929}$, and the nearest $10$-bit
  binary value to $10^{1929}$ is $2^{6408} - 2^{6398}$.

  There are other, smaller values in $\mathbf B$ that fail to roundtrip: the
  first such value larger than $1$ is $1017\cdot 2^{774}$, which lies in the
  interval $(10^{236}, 2^{784})$; values from $\mathbf D$ have a spacing of
  $10^{233}$ in this interval, while values from $\mathbf B$ are spaced
  $2^{774}$ apart; we have $10^{233} / 2^{774} = 1.006429...$.

  For floats larger than $1$, the \emph{first} interval where floats from
  $\mathbf B$ are spaced more closely than floats from $\mathbf D$ is
  $(10^{31}, 2^{103})$, but it turns out that there are no roundtrip
  counterexamples in this interval.
\end{example}

To complete the picture, we're missing one special case.

\begin{proposition}
  Suppose that $B$ and $D$ are powers of a common base $C$, and choose $C$ to
  be the maximal possible common base. The the floating-point format $\mathbf
  B$ roundtrips through $\mathbf D$ if and only if
  $$B^p \le C D^{q-1}.$$
\end{proposition}

\begin{proof}
  For sufficiency, we prove something more, namely that if $$B^p \le C
  D^{q-1}$$ then every element $x$ of $\mathbf B$ is already an element of
  $\mathbf D$. Then roundtripping follows immediately.

  Given an element $x$ of $\mathbf B$, we can write
  $$x = mB^e$$
  for some positive integer $m < B^p$ and integer $e$. Now choose $f$ such
  that
  $$D^f \le B^e < D^{f+1},$$ then we can write $$x = (mB^eD^{-f})D^f.$$ To show
  that $x$ is representable in $\mathbf D$, we need to show that the
  coefficient $mB^eD^{-f}$ of $D^f$ is an integer, and that it's strictly less
  than $D^q$. It's an integer because $B^eD^{-f}$ is a power of $C$ that's
  larger than $1$. To see the bound, first note that our choice of $f$ ensures
  that $CB^e \le D^{f+1}$. Now
  \begin{align*}
    mB^eD^{-f}
    & < B^{p+e}D^{-f} \\
    & \le CD^{q-1}B^eD^{-f} \\
    & \le D^{f+1}D^{q-1}D^{-f} \\
    & = D^q,\\
  \end{align*}
  as required.

  For the converse, suppose that $B^p > CD^{q-1}$. Choose integer exponents $e$
  and $f$ such that $B^e / D^f = C$, and consider the interval $[D^f,
    B^e]$. The endpoints of the interval are representable in both formats
  simultaneously, so the interval maps to itself under rounding in either
  direction.

  So to show that roundtripping fails, it's enough to show that there are more
  $\mathbf B$-floats in this interval than $\mathbf D$-floats, or equivalently,
  that the $\mathbf B$-floats are denser than the $\mathbf D$-floats.

  But the spacing between any two adjacent $\mathbf B$-floats in this interval
  is $B^{e-p}$, while the spacing between any two adjacent $\mathbf D$-floats
  is $D^{f-q+1}$, and we have:
  \begin{align*}
    B^{e-p}
    &= CD^fB^{-p} \\
    &< D^{1-q}D^f \\
    &= D^{f-q+1}.
  \end{align*}
  This completes the proof.
\end{proof}



\begin{example}
  In the special case where $B=D$, we have $C = B = D$, and the proposition
  above tells us that $\mathbf B$ roundtrips through $\mathbf D$ if and only if
  $p\le q$, which is exactly what we'd expect.
\end{example}

\end{section}

\section*{References}
\begin{enumerate}
  \renewcommand\labelenumi{[\theenumi]}
  \item \label{so1} ``How to find an original string representation for lower
    precision float values in Python?'',
    \url{http://stackoverflow.com/a/35708911/270986}
\end{enumerate}

\end{document}
