\documentclass{article}
\usepackage{amsmath}
\usepackage{amsthm}
\usepackage{hyperref}

\theoremstyle{plain}
\newtheorem{lemma}{Lemma}
\newtheorem{proposition}[lemma]{Proposition}
\newtheorem{fact}[lemma]{Fact}

\theoremstyle{definition}
\newtheorem{definition}[lemma]{Definition}
\newtheorem{example}[lemma]{Example}

\begin{document}

This note is a followup from a discussion with Rick Regan in the comments
section of a \href{http://stackoverflow.com/a/35708911/270986}{StackOverflow
  question}. The aim is to give a short proof of the condition for
roundtripping of conversions between two floating-point formats.

\begin{section}{Preliminaries}

We'll work with an idealised form of floating-point that ignores
exponent bounds, NaNs, infinities and zeros. Roundtripping for negative floats
doesn't behave substantially differently than for positive floats, so we also
ignore negative floats and just work with positive floats. We'll only consider
round-to-nearest rounding modes between the two formats, and we won't make any
assumptions about which direction ties round in.

To be precise, here's what we'll mean by a floating-point number in what
follows.

\begin{definition}
  For integers $p \ge 1$ and $B \ge 2$, a \textbf{precision-$p$ base-$B$
    floating-point number} is a positive rational number $x$ that can be
  expressed in the form $m B^e$ for some integers $m$ and $e$, with $0 \le m <
  B^p$.
\end{definition}

We're interested in what happens when we convert a number in one floating-point
format to another format and back again, with each of the two conversions
rounding to nearest. Specifically, we want to give necessary and sufficient
conditions on the two formats for this double conversion to recover the number
we started with.

For the remainder of this note, we fix two floating-point formats.  We choose
integers $p \ge 1$, $q \ge 1$, $B \ge 2$ and $D\ge 2$. Write $\mathbf B$ for
the set of all precision-$p$ base-$B$ floats, and $\mathbf D$ for the set of
all precision-$q$ base-$D$ floats.

\begin{definition}
  We will say that a precision-$p$ base-$B$ float $x$ in $\mathbf B$
  \textbf{roundtrips through $\mathbf D$} if the nearest precision-$q$ base-$D$
  floating-point value to $x$ rounds back to $x$. Or in other words, let $y$ be
  the closest element of $\mathbf D$ to $x$. Then we require that $x$ is the
  closest element of $\mathbf B$ to $y$. (In the event that $x$ is exactly
  halfway between two elements of $\mathbf D$, we'll require that \emph{both}
  those elements round back to $x$.)

  We'll also say that the \emph{format} $\mathbf B$ roundtrips through $\mathbf
  D$ if every element of $\mathbf B$ roundtrips through $\mathbf D$.
\end{definition}

\end{section}

\begin{section}{Roundtrip results}

Now we can state the main propositions. There's a simple sufficient condition
for roundtripping.

\begin{proposition}
  \label{roundtrip_sufficient_condition}
  If $B^p \le D^{q-1}$ then the format $\mathbf B$ roundtrips through $\mathbf
  D$.
\end{proposition}

Under one additional hypothesis, this condition is also necessary.

\begin{proposition}
  \label{roundtrip_necessary_condition}
  If the format $\mathbf B$ roundtrips through $\mathbf D$, \emph{and} $B$ and
  $D$ are not powers of a common base, then $B^p \le D^{q-1}$.
\end{proposition}

The \emph{powers of a common base} condition excludes cases like $B = 4$ and $D
= 8$, or $B = 5$ and $D = 25$, for example.

To prove the first proposition, we need a pair of lemmas relating to the
spacing between successive floating-point numbers.

\begin{lemma}
  \label{small_gap}
  Suppose that $x$ is an element of $\mathbf B$, and that $B^{e-1} < x \le
  B^e$. Then any positive rational number $y$ satisfying $|x - y| < \frac12
  B^{e-p}$ rounds back to $x$.
\end{lemma}

\begin{proof}
  This follows from the observation that within the closed interval $[B^{e-1},
    B^e]$, successive floats in $\mathbf B$ are spaced exactly $B^{e-p}$ apart
  from one another. Some care must be taken with the corner case where $x=B^e$,
  where the next float up from $x$ is $B^e + B^{e-p+1}$ rather than $B^e +
  B^{e-p}$. In particular, note that if we'd stated the condition on $e$ in the
  form $B^{e-1} \le x < B^e$, the conclusion of the lemma would be false.
\end{proof}

\begin{lemma}
  \label{large_gap}
  Suppose that $x$ is any positive rational number, that $f$ is the unique
  integer such that $D^{f-1} < x \le D^f$, and that $y$ is a closest element of
  $\mathbf D$ to $x$, choosing either possibility in the case of a tie. Then
  $|x - y| \le \frac12 D^{f-q}$.
\end{lemma}

\begin{proof}
  This follows from the fact that $D^{f-1} \le y \le D^f$, and that successive
  floats in $\mathbf D$ are spaced exactly $D^{f-q}$ apart in this interval.
\end{proof}

The proof of the sufficient condition is now straightforward.

\begin{proof}[Proof of Proposition \ref{roundtrip_sufficient_condition}]
  Let $x$ be any precision-$p$ base-$B$ floating-point number, and let $y$ be a
  closest precision-$q$ base-$D$ floating-point number to $x$ (picking either
  one in the case of a tie).  Choose integers $e$ and $f$ such that $B^{e-1} <
  x \le B^e$ and $D^{f-1} < x \le D^f$. Then we have:
  \begin{align*}
    |x - y|
    &\le \frac12 D^{f-q} && \text{by Lemma \ref{large_gap}}\\
    &= \frac12 D^{f-1} D^{1-q} \\
    &< \frac12 x D^{1-q} && \text{by choice of $f$}\\
    &\le \frac12 x B^{-p} &&
                   \text{from the assumption that $B^p \le D^{q-1}$}\\
    &\le \frac12 B^e B^{-p} && \text{by choice of $e$}\\
    &= \frac12 B^{e-p} \\
  \end{align*}
  So $|x - y| < \frac12 B^{e-p}$, and it follows from Lemma \ref{small_gap}
  that $y$ rounds back to $x$.
\end{proof}

For the necessary condition, we'll need the following result from elementary
number theory.

\begin{fact}\label{dense_power_ratios}
  Suppose that $B$ and $D$ are integers larger than $1$, and that $B$ and $D$
  are not powers of a common base (or equivalently, $log(B) / log(D)$ is not
  rational). Then numbers of the form $B^e / D^f$ are \emph{dense} in the
  positive reals: any open subinterval $(a, b)$ of the positive reals contains
  at least one such number.
\end{fact}

\begin{proof}
  Take logs to base $D$ and apply Kronecker's approximation theorem.
\end{proof}

Now we can prove the necessary condition.

\begin{proof}[Proof of Proposition \ref{roundtrip_necessary_condition}]
  We prove the contrapositive: that if $B^p > D^{q-1}$ then there's at least
  one element $x$ of $\mathbf B$ that fails to roundtrip through $\mathbf D$.

  To provide some intuition: we're looking for a region of the positive reals
  where the gap between successive elements of $\mathbf B$ is \emph{smaller}
  than the gap between successive elements of $\mathbf D$. In relative terms,
  the gap between successive elements of $\mathbf B$ is smallest just
  \emph{before} a power of $B$, while the gap between successive elements of
  $\mathbf D$ is largest just \emph{after} a power of $D$. So if there's an $x$
  that fails to roundtrip, a good place to look for it would be in an interval
  $[D^f, B^e]$ where $D^f$ and $B^e$ are very close to one another. In such an
  interval, the gap between successive elements of $\mathbf D$ is $D^{f-q+1}$,
  while the gap between successive elements of $\mathbf B$ is $B^{e-p}$. Our
  hypothesis that $B^p > D^{q-1}$ implies that $1 < B^p / D^{q-1}$, so by Fact
  \ref{dense_power_ratios} we should be able to find $e$ and $f$ so that $1 <
  B^e / D^f < B^p / D^{q-1}$. Then our gaps satisfy $B^{e-p} < D^{f-q+1}$, as
  required.

  Making the above rigorous is a tiny bit fiddly: if we make $D^f$ \emph{too}
  close to $B^e$, then the interval $(D^f, B^e)$ is too small to contain any
  floats, making it hard to find a suitable $x$. So we'll also need to give a
  lower bound on how close $D^f$ and $B^e$ get.

  It turns out that we can always find a roundtrip failure of the form $x =
  B^e$: we'll arrange that $x$ rounds down to an exact power of $D$, $D^f$, and
  then that \emph{that} power of $D$ in turn rounds \emph{down} again to
  something smaller than $B^e$, breaking roundtripping.

  For $B^e$ to round down to $D^f$, we need:
  $$D^f < B^e < D^f + \frac12 D^{f-q+1},$$ while for $D^f$ to round down to
  some value strictly smaller than $B^e$, we need
  $$D^f < B^e - \frac12 B^{e-p}.$$
  Rearranging the inequalities above gives the condition
  $$\frac{1}{1 - \frac12 B^{-p}} < B^e / D^f < 1 + \frac12 D^{1-q}.$$ So
  \emph{if} we can find integer exponents $e$ and $f$ such that the above
  holds, then $x=B^e$ gives us a counterexample to roundtripping.

  But from Fact \ref{dense_power_ratios}, we can find a number of the form $B^e
  / D^f$ in \emph{any} open subinterval of the positive reals. The only thing
  we need to check is that the two inequalities above are compatible; that is,
  that
  $$\frac{1}{1 - \frac12 B^{-p}} < 1 + \frac12 D^{1-q},$$
  so that the lower and upper bounds really do form a subinterval of the reals.
  But the above inequality can be rearranged to the equivalent inequality
  $$\frac12 < B^{p} - D^{q-1},$$
  which is true from our assumption that $B^p > D^{q-1}$.

  So this completes the proof: find $e$ and $f$ satisfying the above
  inequality, and then $x = B^e$ will fail to roundtrip.
\end{proof}

\begin{example}
  Consider precision-$53$ binary (as used in double-precision IEEE 754
  floating-point, for example). We have
  $$2^{53} \le 10^{17 - 1}$$ and it follows that conversion from a finite IEEE
  754 binary64 float to a decimal string with $17$ significant digits produces
  a value that rounds back to the original float. $16$ significant digits are
  insufficient.
\end{example}

\begin{example}
  Consider conversions from $10$-bit binary to $4$-digit decimal and back
  again.  The necessary and sufficient condition for roundtripping is that
  $$2^{10} \le 10^{4-1},$$ which is false (but only just). So roundtripping
  does not hold, and we should be able to find an explicit $10$-bit binary
  float which fails to round-trip through $4$-digit decimal.

  Following the above proof, we look for exponents $e$ and $f$ such that
  $$\frac{1}{1 - \frac12 2^{-10}} < \frac{2^e}{10^f} < 1 + \frac12 10^{-3}$$
  which simplies to the condition
  $$\frac{2048}{2047} < \frac{2^e}{10^f} < \frac{2001}{2000}.$$ The smallest
  positive exponents satisfying this pair of inequalities are $f=1929$ and
  $e=6408$ (in fact, this $f$ is the \emph{only} positive value smaller than
  $10000$ for which there's a solution). So $x = 2^{6408}$ fails to roundtrip:
  the nearest $4$-digit decimal to $x$ is $10^{1929}$, and the nearest $10$-bit
  binary value to $10^{1929}$ is $2^{6408} - 2^{6398}$.
\end{example}

To complete the picture, we're missing one special case. I'll state it here
without proof for now.

\begin{proposition}
  Suppose that $B$ and $D$ are powers of a common base $C$, and choose $C$ to
  be the maximal possible common base. The the floating-point format $\mathbf
  B$ roundtrips through $\mathbf D$ if and only if
  $$B^p \le C D^{q-1}.$$
\end{proposition}

\begin{example}
  In the special case where $B=D$, we have $C = B = D$, and the proposition
  above tells us that $\mathbf B$ roundtrips through $\mathbf D$ if and only if
  $p\le q$, which is exactly what we'd expect.
\end{example}

\end{section}

\end{document}
